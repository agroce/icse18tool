\documentclass[sigconf]{acmart}


% Copyright
%\setcopyright{none}
%\setcopyright{acmcopyright}
%\setcopyright{acmlicensed}
\setcopyright{rightsretained}
%\setcopyright{usgov}
%\setcopyright{usgovmixed}
%\setcopyright{cagov}
%\setcopyright{cagovmixed}


% DOI
\acmDOI{10.475/123_4}

% ISBN
\acmISBN{123-4567-24-567/08/06}

%Conference
\acmConference[ICSE'18]{International Conference on Software Engineering}{May-June
  2018}{Gothenburg, Sweden} 
\acmYear{2018}
\copyrightyear{2018}

\acmPrice{15.00}


%\usepackage{booktabs}
%\usepackage{multirow}
%\usepackage{siunitx}
%\usepackage{graphicx}
%\usepackage{textcomp}
%\usepackage{cite}
\usepackage{url}
\usepackage{code}

\newcommand{\mycomment}[1]{}

%\sisetup{round-mode=places, round-precision=2}

%\def\tilde{\texttildelow}

\title{An Extensible Regexp-Based Tool for Multi-Language\\ Mutation Generation }

%\author{
%}

\begin{abstract}
  Mutation testing is widely used in research (even if not in
  practice).  Mutation testing tools usually target only one
  programming language, and rely on parsing a program to generate
  mutants, or operate not at the source level but on compiled
  bytecode.  Unfortunately, developing a robust mutation testing tool
  for a new language in this paradigm is a difficult and
  time-consuming undertaking.  Moreover, bytecode/intermediate
  language mutants are difficult for programmers to read and
  understand.  This paper presents a simple tool, called {\tt
    universalmutator}, based on regular-expression-defined transformations of
  source code.  The primary drawback of such an approach is that our
  tool can generate invalid mutants that do not compile, and sometimes
  fails to generate mutants that a parser-based tool would have
  produced. Additionally, it is incompatible with some approaches to
  improving the efficiency of mutation testing.  However, the
  regexp-based approach provides multiple compensating advantages.
  First, our tool is easy to adapt to new languages; e.g., we present
  here the first mutation tool for Apple's Swift programming
  language.  Second, the method makes handling multi-language programs
  and systems simple, because the same tool can support every
  language.  Finally, our approach makes it easy for users to add
  custom, project-specific mutations.
%{\bf VIDEO AT: \url{https://youtu.be/2T23mPVq2Hs}}
\end{abstract}

% LocalWords:  bytecode universalmutator multi


\begin{document}


\begin{CCSXML}
<ccs2012>
<concept_id>10011007.10011074.10011099.10011102.10011103</concept_id>
<concept_desc>Software and its engineering~Software testing and debugging</concept_desc>
<concept_significance>500</concept_significance>
</concept>
</ccs2012>
\end{CCSXML}


\ccsdesc[500]{Software and its engineering~Software testing and debugging}

\keywords{mutation testing, multi-language tools, regular expressions}


\maketitle




\section{Introduction}

Mutation testing \cite{PracProg,Mut2000} is a widely used technique,
originally proposed as a way for test engineers to understand
omissions in their testing, and now routinely used to evaluate
novel software testing methods \cite{ISSTA13,ahmed_testedness}, and in software
fault localization \cite{Metallaxis,multilingual,TransProgTest}.

Unfortunately, despite the considerable utility of mutation testing,
effective mutation testing tools are surprisingly rare.  Java alone
has widely available, easily used mutation tools, but the most popular
and useful of these, PIT \cite{pittest} operates only at the byte code
level, making it difficult for users to read and understand the
generated mutants.  While there are multiple tools for C mutation,
they often crash on surprisingly simple programs, and both Milu
(\url{https://github.com/yuejia/Milu}) and Proteum
(\url{https://github.com/magsilva/proteum}) have not been updated
since mid-2016.  In fact, the simple Prolog-based tool developed by
Jamie Andrews \cite{mutant} is still widely used in publication and
practice, despite not even working with recent versions of the Prolog compiler.

The {\tt universalmutator} \footnote{\url{https://github.com/agroce/universalmutator}} is a tool for
mutation testing \cite{PracProg,Mut2000} that aims to expand the
applicability of mutation testing by making it easy to extend
effective mutant generation to new programming languages.
The tool's design is informed by four basic principles:

\begin{enumerate}
\item In practice, most source-code mutants of interest can be
  produced without actually parsing the language of a source code
  file.  Most useful mutations are approximately equivalent to a
  set of string manipulations that can be defined by a regular
  expression to match and a replacement string.  This makes it
  possible to avoid the onerous work of building a parser for every
  language to be mutated.
\item Modern optimizing compilers are better at detecting invalid and
  (more importantly) equivalent and redundant mutants than any tool
  that can be produced with a reasonable amount of effort.  Reducing
  the set of mutants to be tested to useful mutants is most easily
  approached by leveraging Trivial Compiler Equivalence \cite{TCE}.
\item Because of the theoretical and practical limits of mutation
  reduction strategies \cite{Gopinath1,Gopinath2}, the
  effectiveness of random sampling \cite{Gopinath3}, and the need for
  novel mutants to correlate to important classes of software defect
  \cite{Just2014mutants}, it is essential for a mutation tool to make adding new
  kinds of mutations easy, and to support custom, project-specific
  mutants defined by developers expert in their own code base.
\item Mutants are used in a wide variety of contexts; while it limits
  the efficiency of our tool, we aim to make it easy to integrate it
  into a variety of testing work-flows and software engineering
  algorithms.  This leads us to a UNIX-like ``little tools''
  philosophy, where rather than having one monolithic program for
  mutant generation, execution, and presentation of results, we
  provide separate tools for mutant generation, mutant execution,
  exclusion of mutants based on code coverage, and so forth, using
  simple text files as a means of communication between these tools.
\end{enumerate}

\begin{figure}
{\scriptsize
\begin{code}
DO\_NOT\_MUTATE ==> DO\_NOT\_MUTATE

\\+ ==> -
\\+ ==> *
\\+ ==> /
\\+ ==> \%

-([^>]) ==> +\\1
-([^>]) ==> *\\1
-([^>]) ==> /\\1
-([^>]) ==> \%\\1

([^/*])\*([^/*]) ==> \\1+\\2
([^/*])\*([^/*]) ==> \\1-\\2
([^/*])\*([^/*]) ==> \\1/\\2
([^/*])\*([^/*]) ==> \\1\%\\2

([^\*/])/([^\*/]) ==> \\1+\\2
([^\*/])/([^\*/]) ==> \\1-\\2
([^\*/])/([^\*/]) ==> \\1*\\2
([^\*/])/([^\*/]) ==> \\1\%\\2

\% ==> +
\% ==> -
\% ==> *
\% ==> /

!= ==> ==
!= ==> <=
!= ==> >=
!= ==> >
!= ==> <

== ==> !=
== ==> <=
== ==> >=
== ==> >
== ==> <

>= ==> ==
>= ==> !=
>= ==> <
>= ==> >

<= ==> ==
<= ==> !=
<= ==> <
<= ==> >

< ==> >
< ==> ==

([^-])> ==> \\1<
([^-])> ==> \\1==

-([^>]) ==> \\1

(\\D)(\\d+)(\\D) ==> \\g<1>0\\3
(\\D)(\\d+)(\\D) ==> \\g<1>1\\3
(\\D)(\\d+)(\\D) ==> \\g<1>-1\\3
(\\D)(\\d+)(\\D) ==> \\1\\2+1\\3
(\\D)(\\d+)(\\D) ==> \\1\\2-1\\3
\&\& ==> ||
\\|\\| ==> \&\&
! ==>

([^\&])\&([^\&]) ==> \\1|\\2
([^|])\\|([^|]) ==> \\1\&\\2

(^\\s*)(\\S+.*)\\n ==> \\1\\2\\n\\1break;\\n

".+" ==> ""
\end{code}
}
\caption{Universal mutation rules for all languages.}
\label{fig:universal}
\end{figure}

Such an approach has limitations, of course.  In general, it will
usually produce more invalid mutants, or generally uninteresting mutants (e.g.,
modifying part of a constant string), and it will be difficult to
apply many of the methods proposed for speeding up mutation testing,
such as mutant schemata \cite{untch1993mutation} or ``forking''
approaches \cite{Equiv,TopsyTurvy}.  However, we believe that having a
somewhat inefficient mutation tool is better, for the many languages
currently lacking tools, than having no tool at all.  Moreoever, the
more sophisticated techniques for making mutation testing more
efficient also make it more difficult to allow custom mutation
operators, discouraging project-specific approaches or experimentation
with novel operators for, e.g., concurrency or functional programming.

The design of {\tt universalmutator} is simple.  After determining
which language a source file is written in, it applies appropriate
regular expression based mutation rules, contained in {\tt .rules}
files (a number of which are provided with the tool).  In most cases, a source file will be mutated based on multiple
rule files:  there is a set of ``universal'' mutations rules, common
to all languages, a set of rules for C-like languages, and then
(often) as set of rules for each specific language.  For instance,
mutating a Swift program uses the {\tt .rules} files {\tt
  universal}, {\tt c\_like}, and {\tt swift}.  

Figure \ref{fig:universal} shows the set of rules that are applied in
common to all languages.  Some of these rules are irrelevant to some
languages (for instance, Python does not use {\tt \&\&} as a logical
operator), but in such cases the presence of the rule is harmless.  The
core mutations defined here include various operator replacements,
numeric and string constant replacements (introducing several
additions to the set used by Andrews \cite{mutant}, such as replacing
any string by the empty string), and the insertion of new {\tt break}
statements.  The universal rules also include an example of a
mechanism for annotating that a line should never be mutated.  This is
a feature not (to our knowledge) present in most tools, but can be
useful:  first, some lines may be known to related to untested code
(e.g., {\tt WARNING} messages that are never checked by tests);
second, in rare instances a statement may be dangerous to mutate, such
as a network or file system operation, but there is interest in
mutating other aspects of computation.  The example rule file makes
the structure of a mutation rule clear:  the left hand side of a
mutation is a regular expression (regexp), using Python's notation for regexps;
the right hand side (across the {\tt ==>}) is a Python regular
expression replacement string, which can use {\tt \textbackslash 1}, {\tt
  \textbackslash 2}
etc. to include groups from the regexp that matched.  Under this
scheme, most mutations are simple to define, and the only major
limitation for common mutation operators is that the approach does not
handle deletion of multiple-line statements.





\bibliographystyle{ACM-Reference-Format}
\bibliography{tool}

\end{document}
