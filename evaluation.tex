\section{Evaluation}

As a simple evaluation of the capabilities of our approach, we compared the mutants generated by {\tt
  universalmutator} to those generated by Andrews' tool \cite{mutant},
used in papers on mutation-based model checking of C programs
\cite{ASE15} and mutations as a tool for improving Linux kernel test
suites \cite{mutation17}.  The mutants generated by our tool were a
superset of those generated by Andrews' tool for the 8 C files used in
these papers.  Most importantly, the actual mutants used to drive
fixes to the model-checking harnesses or the Linux kernel test suite
were produced by both tools.

We also applied our tool to an ongoing effort to test the {\tt
  pyfakefs} file system \cite{pyfakefs}, using TSTL \cite{nfm15,tstlsttt},
and confirmed that our mutants were equally useful for test suite
evaluation and improvement as the mutants produced by a much more
complex, AST-based tool, {\tt muupi} \cite{muupi}.
\begin{table}[t!]
  \scriptsize
  \caption{\label{tab:java} Java mutation results}
  \begin{tabular}{|l|lll|lll|lll|}
    \hline
    Subjects&\multicolumn{3}{c|}{PIT}&\multicolumn{3}{c|}{Major}&\multicolumn{3}{c|}{{\tt universalmutator}}\\
    &Gen&Kill&MS&Gen&Kill&MS&Gen&Kill&MS\\\hline
\hline
    {\tt Triangle}&45&44&97.78\%&130&130&93.53\%&188&184&97.87\%\\
{\tt FizzBuzz}&111&92&82.88\%&244&202&82.79\%&203&176&86.70\%\\
  \hline\end{tabular}
  \end{table}

Finally, we compared our results to PIT (Version 1.2.5-SNAPSHOT)
\cite{pittest} and Major (Version v1.3.2) \cite{major} for both the
classic {\tt Triangle} example and a real-world GitHub project, {\tt
  FizzBuzz}\footnote{\url{https://github.com/EnterpriseQualityCoding/FizzBuzzEnterpriseEdition}
}.  Note that we used all the default mutation operators for both
Major and PIT. To illustrate, here are the commands that we used to
perform mutation testing using {\tt universalmutator} for {\tt
  Triangle}:
{\scriptsize
\begin{verbatim}
% mkdir mutants
% mutate src/main/java/triangle/Triangle.java --mutantDir mutants
...
% analyze_mutants src/main/java/triangle/Triangle.java "mvn test"
  --mutantDir mutants
\end{verbatim}
} The experimental results are shown in Table~\ref{tab:java}. For each
tool on each subject, we show the number of generated mutants
(Column ``Gen''), the number of killed mutants (Column ``Kill''), and
the mutation score (Column ``MS''). From the table, we can observe that all three tools provide consistent mutation scores, demonstrating the validity of our tool for Java, even without the ability to parse Java (or any Java-specific rules beyond the erasure of the {\tt synchronized} keyword) or handle multi-line constructs.

\mycomment{For {\tt Triangle},
  PIT generates 87 mutants, of which 85 are killed for a 97.70\%
  mutation score.  Major generates 139 mutants, of which 130 are
  killed for a mutation score of 93.53\%.  The {\tt universalmutator}
  using the following commands, generates 188 mutants, of which 184
  are killed for a mutation score of 97.87\%:}
