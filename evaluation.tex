\section{Evaluation}

As a simple evaluation of the capabilities of our approach, we compared the mutants generated by {\tt
  universalmutator} to those generated by Andrews' tool \cite{mutant},
used in papers on mutation-based model checking of C programs
\cite{ASE15} and mutations as a tool for improving Linux kernel test
suites \cite{mutation17}.  The mutants generated by our tool were a
superset of those generated by Andrews' tool for the 8 C files used in
these papers.  Most importantly, the actual mutants used to drive
fixes to the model-checking harnesses or the Linux kernel test suite
were produced by both tools.

We also applied our tool to an ongoing effort to test the {\tt
  pyfakefs} file system \cite{pyfakefs}, using TSTL \cite{nfm15,tstlsttt},
and confirmed that our mutants were equally useful for test suite
evaluation and improvement as the mutants produced by a much more
complex, AST-based tool, {\tt muupi} \cite{muupi}.

Finally, we compared our results to PIT \cite{pittest} and Major
\cite{major} for the classic {\tt Triangle.java} example.  PIT
generates 45 mutants, of which 44 are killed for a 97.78\% mutation
score.  Major generates 139 mutants, of which 130 are killed for a
mutation score of 93.53\%.  The {\tt universalmutator} using the
following commands, generates 188 mutants, of which 184 are killed for
a mutation score of 97.87\%:

{\scriptsize
\begin{verbatim}
% mkdir mutants
% mutate src/main/java/triangle/Triangle.java --mutantDir mutants
...
% analyze_mutants src/main/java/triangle/Triangle.java "mvn test"
  --mutantDir mutants
\end{verbatim}
}