\section{Example Usage}

The tool can easily be installed using Python's {\tt pip} package management system:

\begin{code}
\% pip install universalmutator
\end{code}

The module is pure Python and depends on no modules other than
built-in Python libraries, so should be usable on almost any system
that supports Python 2.7+.  Mutation with TCE of a particular
language, of course, may depend on having that language's compiler
available (e.g., {\tt swiftc} is used to compile swift code).

To mutate a source file, simply type:


\begin{code}
\% mutate <sourcefile>
\end{code}

If the source file extension is one of those supported (currently {\tt
  .c}, {\tt .cpp}, {\tt .java}, {\tt .py}, and {\tt .swift}), the tool
will automatically determine the rules to use and generate a set of
mutants.  For example, if {\tt <sourcefile>} is {\tt avltree.c}, it
will generate {\tt avltree.mutant.0.c}, {\tt avltree.mutant.1.c}, and
so forth.  For Swift and Python files, the tool will also attempt to
compile each mutant, and only generate mutant files for mutants that
(1) compile without syntax errors and (2) do not produce equivalent
object files to either the original source code or another
already-generated mutant.  Because the use of a linker, include paths, and
compiler directives makes it difficult to automatically check C and
C++ files for successful compilation\footnote{This difficulty is a
  significant contrast to languages
  like Python or Swift, where even a file that is part of a large
  project is often easily re-compiled by itself, without any extra
  options or configuration.}, the tool also supports a mode where the
user provides a compilation command to be used to check for valid
mutants, e.g.:


\begin{code}
\% mutate foo.c --cmd ``clang -c MUTANT -I .''
\end{code}

\noindent where {\tt MUTANT} will be replaced by the name of a
generated temporary file for the mutant, clang will be used as the
compiler, and the current directory will be used as an include path.  Constructing a TCE checker
to compare C and C++ object files is non-trivial, due to build
complexities and object formats that include timestamps, so we allow
users to write Python handlers that perform that task in a project and
compiler specific-way.

To analyze the generated mutants, just tell the tool the name of the
source file to be mutated, and the command to run to process the
mutants.  For example, the complete process for performing mutation
analysis on the TSTL \cite{tstlsttt,nfm15} Python testing tool's AVL tree example from scratch is just:

{\scriptsize
\begin{verbatim}
% pip install tstl
% git clone https://github.com/agroce/tstl.git
% cd tstl/examples/AVL
% tstl avlnew.tstl
% mutate avl.py
% analyze_mutants avl.py ``ulimit -t30; tstl_rt -t 10''
\end{verbatim}
}

The {\tt mutate} tool will generate 793 valid mutants, 488 invalid
mutants (e.g., changing an {\tt if} into a {\tt pass} in a failed
statement deletion), and 19 redundant mutants equivalent either to one
of the valid mutants or the original {\tt avl.py}.
Running {\tt analyze\_mutants} will produce a pair of files, {\tt notkilled.txt} and {\tt
  killed.txt}, containing names of mutants for which the {\tt tstl\_rt}
command (which runs TSTL's random tester, with a timeout of 10
seconds, here run with a ulimit of 30 seconds to catch mutants
producing non-termination)
returned an error code of zero or non-zero, respectively.  Ten seconds
of random testing kills 388 of the mutants, but 405 survive.

The {\tt analyze\_mutants} tool optionally takes a third argument, the
name of a file containing mutants to ignore.  This lets users write
their own code to prune mutants without having to remove the mutant
files.  The {\tt universalmutator} provides an additional tool, {\tt
  check\_covered <sourcefile> <coveragefile> <outfile>} that scans all
mutants of sourcefile, determines their location (recall that for us all mutants will correspond to a single line
of {\tt sourcefile}), and then outputs a file ({\tt outfile})
containing all mutants not present in {\tt coveragefile}.  The
expected format for {\tt coveragefile} is a list of line numbers,
delimited by whitespace, starting from 1; the optional {\tt --tstl}
argument also allows the tool to process internal coverage reports
from the {\tt tstl} testing tools.  Of the 405 surviving mutants, 67
are in code not covered during even 5 minutes of random testing, mostly test code not part of the actual AVL class.

\section{Extending the Mutator}

The parser-free nature of {\tt universalmutator} makes it easy to
use in ways that might not easily
work with a more traditional mutation tool.  For example, it can
handle files that are ``almost'' in some known-to-the-tool language
just as well as files actually in the language, which would usually be
rejected by a parser-based tool.  Model checking and
testing tools, such as TSTL  and Spin \cite{SPIN}, embed other
languages in test harnesses (Python for TSTL, C for Spin's PROMELA
language).  In these cases you just mutate the appropriate file, after
telling the mutator what language it is ``written in.''  Some mutants
may be invalid, but these will just be rejected by the language tools
and discarded.

However, in some cases simply using an existing language's rules to
mutate something close to that language is not
all that is needed.  You may want to add a new mutant, for a project,
or you may need to define basic mutation operators for a novel
programming language with unusual syntax.  In both cases, the solution
is simple:  just write a new {\tt .rules} file, and either add it to
the {\tt mutate} call, or, if you really need to go outside the box,
and the {\tt univeral} mutation rules no longer apply, use the
language {\tt none} and explicitly note the rule file(s) to be used.

Because of the regexp style of defining operators, you can write a
quick, clumsy rule if you want to try an experimental operator out and
see how well it works, or refine it further.  For instance, one might
think that changing {\tt while} loops into {\tt if} statements is a
promising idea for a mutation operator\footnote{We added this one to
  the universal rules file, after writing about it here.}.  This doesn't represent a common human error,
probably, but can check test suites to make sure they are good at
investigating loop iterations beyond running zero or one or more
times.  A quick-and-dirty sketch of the rule might look like:

\begin{code}
while ==> if 
\end{code}

And this version of the rule will largely suffice.  It will, of
course, replace {\tt while} strings in function names, but perhaps
this isn't enough of a problem to matter.  If you do eventually decide (for instance when
working with a large project where there are dozens of methods on a
timed event class named with the pattenr {\tt whileFoo}) that the rule
needs refining, you can ensure that while is a word on its own:

\begin{code}
while(\\W) ==> if\\1 
\end{code}

\noindent This forces the character after the while to be something
that shows the {\tt while} is not part of a valid identified, in most
languages.  We capture it using the parentheses, and reproduce the
same character in the replacement using {\tt \textbackslash1}.

Another use for the same mechanism is to \emph{avoid} mutating code.
For instance, imagine testing a large project with many warning
messages, always consisting of the use of a {\tt WARNING} macro.  The
system tests do not make use of these warnings, and so they end up as
a large set of equivalent mutants, modifying the strings, deleting the
statements, and modifying numeric constants or arithmetic operators in
the warnings.  Annotating all of these with {\tt DO\_NOT\_MUTATE} is
inconvenient and hurts program readability.  Instead, you can just add a
{\tt nowarnings.rules} to every {\tt mutate} call:

\begin{code}
WARNING ==> DO\_NOT\_MUTATE
\end{code}